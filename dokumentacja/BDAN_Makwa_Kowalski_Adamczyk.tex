%styl klasy z Polskimi Normami oprac. Marcin Wolinski
\documentclass[a4paper,titleauthor]{mwart} 
\usepackage{polski}
\usepackage[utf8]{inputenc}
\usepackage{graphicx} %pakiet do wstawiania grafiki
\usepackage[hyphens]{url} %pakiet do wstawiania linkow
%\usepackage[hidelinks,breaklinks]{hyperref}
\usepackage{authblk}%pakiet do tworzenia afiliacji
\usepackage{tabularx}%pakiet do tabel
\usepackage[a4paper, left=2cm, right=2cm, top=3cm, bottom=3cm]{geometry}
\usepackage{listings}
\usepackage{placeins}%pakiet do kontroli umieszczania obiektow
\usepackage{hyperref}%pakiet do m.in. kolorowania linkow
\renewcommand\figurename{Rys.}%skrocony podpis
\renewcommand\lstlistingname{Wydruk}


%------------------------------------------------------------------------
% Dane do strony tytułowej

\title{Dokumentacja implementacji algorytmu Makwa w języku Python \\ Bezpieczeństwo Danych}

\author{Mikołaj Kowalski\\Witold Adamczyk}

\affil{Politechnika Warszawska, Instytut Telekomunikacji}
\date{\today}

%------------------------------------------------------------------------
% Początek dokumentu
\begin{document}
	
%Automatycznie generowany tytuł dokumentu
\maketitle
%podpis na dole strony, bez numeru


%------------------------------------------------------------------------
% Historia zmian

%------------------------------------------------------------------------
% Automatycznie generowany spis treści
\tableofcontents
%------------------------------------------------------------------------
\section{Wstęp}
Zaimplementowaliśmy następujące funkcje algorytmu Makwa:
\begin{itemize}
	\item \textit{Key-Derivation Function}
	\item Kodowanie i dekodowanie liczb całkowitych
	\item \textit{Pre-Hashing}
	\item \textit{Core Hashing}
	\item \textit{Post-Hashing}
	\item \textit{Unescrow}
	\item Generacja pliku \textit{kat.txt}
	\item Generacja losowego klucza prywatnego
	\item Eksport stanu instancji algorytmu
	\item Delegacja (generowanie, zapis i odczyt parametrów, tworzenie par maskujących)
	\item Zmiana \textit{work factor} (ustawienie nowej wartości/zmiana o pewną wartość)
\end{itemize}
Elementy nie zaimplementowane:
\begin{itemize}
	\item \textit{Fast path}
	\item Generacja \textit{soli}
\end{itemize}
W ramach naszej implementacji stworzyliśmy następujące pliki:
\begin{itemize}
	\item \textit{makwa.py} zawierający klasę Makwa oraz funkcje \textit{KDF, Unescorow, Pre-Hashing, Post-Hashing, Core Hashing} wraz z funkcją eksportującą stan instancji
	\item \textit{makwakeys.py} zawierający klasę MakwaPrivateKey która, wraz z funkcjami pomocniczymi, umożliwia wczytanie klucza prywatnego, klucza publicznego oraz odpowiedni eksport tych kluczy
	\item \textit{encoding.py} zawierający pomocnicze funkcje kodujące i dekodujące liczby oraz formatujące odpowiednio napisy.
	\item \textit{modInverse.py} zawierający funkcję \textit{modInverse($a$, $m$)} umożliwiającą znalezienie liczby $a^{-1}$ takiej że: $$a\cdot a^{-1}=_m1$$
	Implementacja znaleziona pod adresem \url{https://www.geeksforgeeks.org/multiplicative-inverse-under-modulo-m/}
	\item \textit{selftest.py} zawierającą funkcję \textit{check} która była przez nas używana znajdowania błedów w implementacji
	\item \textit{makeKat.py} umożliwiający stworzenie pliku \textit{kat.txt} umożliwiającego sprawdzenie poprawności naszej implementacji.
	\item \textit{deleggen.py} umożliwiający wygenerowanie żądania delegacji i eksport w postaci binarnej
\end{itemize}
Poza testem implementacji za pomocą \textit{makeKat} mikro testy poszczególnych funkcji znajdują się w funkcjach \textit{main()} w plikach \textit{makwakeys.py}, \textit{makwa.py} i \textit{deleggen.py}

Zaimplementowaliśmy również podstawowe komunikaty błędów w przypadku podania błędnych parametrów w niektórych funkcjach

W części funkcji bazowaliśmy głownie na opisie zawartym w specyfikacji, a w innych opieraliśmy się na udostępnionej nam implementacji Makwy w języku Java
\section{Implementacja podstawowych funkcji}
\subsection{Konstrukcja klasy Makwa}
Pola klasy \textit{Makwa}:
\begin{itemize}
	\item n - liczba całkowita, główny parametr uzywany do haszowania hasła (\textit{modulus}) (musi zostac podany)
	\item h - funkcja haszująca używana w danej instancji klasy, nasza implementacja umożliwia wykorzystanie tylko SHA256 lub SHA512. (domyślnie SHA256)
	\item preHashing - \textit{boolean} określający czy ma być przeprowadzony proces Pre-Hashing. (domyślnie \textit{True})
	\item t - liczba całkowita, określająca długość wyniku z funkcji Post-Hashing, gdy t=0 proces ten nie następuje. (domyślnie 0)
	\item w - \textit{work\_factor}, liczba całkowita wpływająca na trudność złamania hasła
	\item mod\_id - ID modulusa generowane jako wynik funkcji kdf(n, 8). Wykorzystywane przy eksporcie stanu instancji.
\end{itemize}
Metody klasy Makwa:
\begin{itemize}
	\item kdf(self, m, s) - przeprowadzająca operację KDF na liczbie m, wykorzystując do tego funkcję haszująca h określonej instancji klasy. Wynik jest rozszerzany lub obcinany do długości s.
	\item pre\_hashing(self, password) - przeprowadza proces Pre-Hashing jeśli pole preHashing zostało ustawione na True
	\item post\_hashing(self, hash) - przeprowadza proces Post-Hashing jeśli pole t zostało na liczbę większą od 0
	\item hash(self, password, salt) - przeprowadza proces Core-Hashing
	\item check(self, ref, hashed) - zwraca wartość wyrażenia ref==hashed, używana w trakcie testów
	\item encode\_output(self, salt, pre\_hash, post\_hash\_len,wf,tau) zwraca napis określający parametry instancji klasy o n równym self.n oraz pozostałych parametrach jak podane. (tau to wynik metody hash)
	Zwrócony napis ma format: \textit{base64(mod\_id)\_xddd\_base64(salt)\_base64(tau)}Gdzie \textit{x} to odpowiedni (określony w specyfkacji) znak definiujący ustawienia Pre- i Post- Hashing, \textit{abc} to liczby określające wartość work factor ($wf=a\cdot 2^{10b + c}$). Zapis \textit{base64(x)} wskazuje zapis napisu (liczby zapisanej heksadecymalnie) jako w formacie o podstawie 64.
	\item set\_new\_wf(self, output, new\_wf) odczytuje napis określający parametry instancji klasy, na podstawie których wykonuje zmianę (w obecnej implementacji tylko na większy) \textit{workfactor}
	i zwraca nowy napis po przeprowadzeniu zmiany \textit{workfactor}.
	\item change\_wf(self, privkey, hash, diff) przeprowadza zmianę \textit{work factor} o określoną wartość. Klucz prywatny \textit{privkey} nie musi być podany, ale jego obecność jest konieczna jeśli zamierza się zmniejszyć wartość \textit{work factor} (ujemna wartość argumentu \textit{diff})
\end{itemize}
\subsection{Konstrukcja klasy MakwaPrivateKey}
Pola klasy \textit{MakwaPrivateKey}:
\begin{itemize}
	\item p, q - liczby całkowite z których jest obliczane n (modulus)
	\item QRGen - generator residuów kwadratowych (nie wykorzystywany przez nas w żadnej z funkcji)
	\item modulus (n) - $n=p\cdot q$
\end{itemize}
Metody klasy MakwaPrivateKey
\begin{itemize}
	\item exportPublic(self) - eksportuje klucz publiczny.
	\item exportPrivate(self) - eksportuje klucz prywatny.
\end{itemize}
Oprócz tego w pliku \textit{makwakeys.py} zaimplementowaliśmy funkcje:
\begin{itemize}
	\item encodePrivate(p, q, QRGen) - tworzy klucz prywatny na podstawie zadanych parametrów.
	\item encodePublic(modulus, QRGen) - tworzy klucz publiczny na podstawie zadanych parametrów.
	\item decodePublic(encoded) - wydobywa z klucza publicznego wartość liczby n (modulus)
	\item makeMakwaPrivateKey(encoded) - tworzy instancje klasy MakwaPrivateKey z zadanego klucza prywatnego
	\item generatePrivateKey(size) - generuje obiekt klasy MakwaPrivateKey stworzony z losowego klucza prywatnego o długości \textit{size} bitów
\end{itemize}
Pozostałe funkcje znajdujące się w pliku są funkcja pomocniczymi stworzonymi dla ułatwienia korzystania z funkcji wyżej wymienionych.
\subsection{Konstrukcja klasy Delegation}
Pola klasy \textit{Delegation}:
\begin{itemize}
	\item modulus
	\item \textit{work factor}
	\item \textit{alpha\_arr}, \textit{beta\_arr}: Zbiory liczb $\alpha$ i $\beta$
	\item \textit{with\_gen} - określa, czy delegacja zawiera generator reszt kwadratowych
\end{itemize}
Metody klasy \textit{Delegation}:
\begin{itemize}
	\item export(self) - tworzy sekwencję bajtów delegacji, zawierającą wszystkie informacje o parametrach
	\item createMaskPair(self) - służy do tworzenia pary maskującej
\end{itemize}
Inne metody w pliku \textit{deleggen.py}
\begin{itemize}
	\item generate(mparam, wf, param\_type) - służy do wygenerowania nowej delegacji na podstawie podanych argumentów:
	\subitem mparam - zakodowany modulus instancji Makwa lub klucz prywatny
	\subitem wf - \textit{work factor}
	\subitem param\_type - Opcja generacji (losowe 300 par liczb, rozszerzenie[obliczenie modulus+64 par liczb], jedna para liczb)
	\item generate\_rnd\_pairs(mkw, wf) - generuje 300 losowych par liczb, po czym wykorzystuje je do utworzenia nowej delegacji
	\item makeDelegation(encoded) - wczytuje wyeksportowaną wcześniej delegację
\end{itemize}
\subsection{KDF}
KDF (\textit{Key-Derivation Function}) została zaimplementowana bardzo dokładnie odwzorowując kroki przedstawione w specyfikacji.
\begin{lstlisting}[language=Python]
def kdf(self, m, s):
	r = 32
	if self.h == sha256:
		r = 32
	if self.h == sha512:
		r = 64
	# 1. V <- 0x01 0x01 0x01 ... 0x01
	V = b'\x01' * r
	# 2. K <- 0x00 0x00 0x00 ... 0x00
	K = b'\x00' * r
	# 3. K < - HMAC_K(V | | 0x00 | | m)
	K = hmac.new(K, msg=(V + b'\x00' + m), digestmod=self.h).digest()
	# 4. V < - HMAC_K(V)
	V = hmac.new(K, msg=V, digestmod=self.h).digest()
	# 5. K <- HMAC_K(V || 0x01 || m)
	K = hmac.new(K, msg=(V + b'\x01' + m), 		digestmod=self.h).digest()
	# 6. V < - HMAC_K(V)
	V = hmac.new(K, msg=V, digestmod=self.h).digest()
	# 7. Set T to an empty sequence.
	T = b''
	# 8. While the length of T is not at least equal to s
	# (the requested output length), do the
	# following: V <- HMAC_K(V), T <- T||V
	while len(T) < s:
		V = hmac.new(K, msg=V, digestmod=self.h).digest()
	T += V
	# The output Hs(m) then consists in the s leftmost bytes of T.
	return T[:s]
\end{lstlisting}
\subsection{Kodowanie liczb}
Kodowanie i dekodowanie liczb (konwersja na zapis heksadecymalny i odwrotnie) bazuje na opisie w specyfikacji, lecz również uwzględnia właściwości języka Python.

Podstawowe funkcje kodujące i dekodujące:
\begin{lstlisting}[language=Python]
# 2.4 Integer Encoding
# I2OSP as well
# Tested: OK
def encode(x, outlen=None):
	ret_x = b''
	while x != 0:
		ret_x = pack('=B', x & 0xff) + ret_x
		x >>= 8
	if outlen and len(ret_x) < outlen:
		ret_x = b'\x00' * (outlen - len(ret_x)) + ret_x
	return ret_x


# To test encoding
# OS2IP as well
# Tested: OK
def decode(ret_x):
	x = 0
	k = len(ret_x)
	for i in range(k):
		x = x + ret_x[i] * pow(2, 8 * (k - 1 - i))
	return x
\end{lstlisting}
Implementacja obu funkcji opiera się wprost na opisie w sekcji \textit{2.4 Integer Encoding} ze specyfikacji.
\subsection{Core Hashing}
Główna funkcja haszująca, podobnie jak implementacja \textit{KDF} opierała się na wprost na opisie w specyfikacji.
\begin{lstlisting}[language=Python,mathescape]
# 2.6 Core Hashing
# Basic Test: OK
def hash(self, password, salt):
	password = self.pre_hashing(password)
	# 1. Let S be the following byte sequence (called the padding):
	# (u = len(password) (in bytes))
	# S = H_(k-2-u)(salt || password || u)
	k = len(encode(self.n))
	u = len(password)
	# u must be such that u $\leq$ 255 and u $\leq$ k - 32
	if u > 255 or u > k - 32:
		raise ValueError('Password is too long')
	S = self.kdf(salt + password + pack('=B', u), k - 2 - u)
	# 2. Let X be the following byte sequence:
	# X = 0x00 | | S | | $\pi$ | | u
	X = b'\x00' + S + password + pack('=B', u)
	# 3. Let x be the integer obtained by decoding X with OS2IP.
	x = decode(X)
	# 4. Compute:
	# y = x^(2^(w+1)) (mod n)
	# This computation is normally performed by repeatedly squaring x modulo n;
	# this is done w + 1 times.
	y = x
	for _ in range(self.w + 1):
		y = pow(y, 2, self.n)
	# 5. Encode y with I2OSP into the byte sequence Y of length k bytes.
	Y = encode(y, k)
	# The primary output of MAKWA is Y
	return self.post_hashing(Y)
\end{lstlisting}
\section{Implementacja zaawansowanych funkcji}
\subsection{Unescrow}
Użycie funkcji \textit{unescrow} w celu odzyskania hasła (porcji danych) z hasza możliwe jest tylko w przypadku nie użycia w czasie haszowania procesów pre\_hashingu i post\_hashingu oraz dostępu klucza prywatnego (wartości $p$ i $q$) użytego do obliczenia haszu.

Podczas przeprowadzenia tego procesu konieczne jest wykorzystanie Chińskiego Twierdzenia o Resztach.

Poniżej nasza implementacja wraz z funkcjami pomocniczymi. Implementacja w dużej mierze oparta na implementacji tej funkcji w języku Java
\begin{lstlisting}[language=Python]
def square_root_mod(d, p):
	return pow(d, (p + 1) / 4) % p


def chinese_remainder(p, q, iq, yp, yq):
	h = ((yp - yq) * iq) % p
	y = yq + (q * h)
	return y


def sqrtExp(p, w):
	e = (p + 1) >> 2
	return pow(e, w, p - 1)
	
	
def unescrow(privateKey, hashed, salt, hashFunction, workFactor):
	mpriv = makeMakwaPrivateKey(privateKey)
	y = decode(hashed)
	p = mpriv.p
	q = mpriv.q
	iq = mpriv.invQ
	ep = sqrtExp(p, workFactor + 1)
	eq = sqrtExp(q, workFactor + 1)
	x1p = pow((y % p), ep, p)
	x1q = pow((y % q), eq, q)
	x2p = (p - x1p) % p
	x2q = (q - x1q) % q
	xc = [0] * 4
	xc[0] = chinese_remainder(p, q, iq, x1p, x1q)
	xc[1] = chinese_remainder(p, q, iq, x1p, x2q)
	xc[2] = chinese_remainder(p, q, iq, x2p, x1q)
	xc[3] = chinese_remainder(p, q, iq, x2p, x2q)
	for i in range(4):
		buf = encode(xc[i], len(encode(mpriv.modulus)))
		k = len(buf)
		if buf[0] != 0x00:
			continue
		u = buf[k - 1] & 0xFF
		if u > (k - 32):
			continue
		tmp = salt + buf[k-1-u:k]
		S = Makwa(mpriv.modulus, hashFunction, False, 0, workFactor).kdf(tmp, /
		 k - u - 2)
		pi = buf[k - 1 - u:k - 1]
		flag = False
		for j in range(len(S)):
			if S[j] != buf[j + 1]:
				flag = True
				break
		if flag:
			continue
		pi = buf[k - 1 - u:k - 1]
		return pi
\end{lstlisting}
\subsection{Generacja klucza prywatnego}
Wzorując się na implementacji algorytmu w języku Java  zaimplementowaliśmy również generacje klucza prywatnego określone długości.

Implementacja ta korzysta z testu Millera-Rabina. Funkcja \textit{computeNumMR} zwraca liczbę rekomendowanej liczby rund koniecznych do przeprowadzenia testu Millera-Rabina zaimplementowanego w funkcji \textit{passesMR}

Funkcja generująca klucz prywatny:
\begin{lstlisting}[language=Python]
def generatePrivateKey(size):
	if size < 1273 or size > 32768:
		raise ValueError('Invalid modulus size: '+str(size))
	k = (size - 14) // 4
	x = 0
	case = (size-14) & 3
	if case == 0:
		x = 7
	elif case == 1:
		x = 8
	elif case == 2:
		x = 10
	else:
		x = 12
	sp = []
	used = []
	bp = []
	length = (k+12) >> 3
	mz16 = 0xFFFF >> (8 * length - k)
	mo16 = x << (k + 16 - 8 * length)
	numMR = computeNumMR(k)
	numMR2 = computeNumMR(k << 1)
	flag = True
	while flag:
		buf = os.urandom(length)
		buf = encode(buf[0] & (mz16 >> 8), 1) + buf[1:]
		buf = encode(buf[0], 1) + encode(buf[1] & mz16,1) + buf[2:]
		buf = encode(buf[0] | (mo16 >> 8), 1) + buf[1:]
		buf = encode(buf[0], 1) + encode(buf[1] | mo16,1) + buf[2:]
		buf = buf[:(length-1)] + encode(buf[length - 1] | 0x01)
		pj = decode(buf)
		if isMultipleSmallPrime(pj):
			continue
		if not passesMR(pj, numMR):
			continue
		flag1=False
		flag2=False
		for z in sp:
			if z == pj:
			flag1 = True
			break
		if flag1:
			continue
		for i in range(len(sp)-1,-1,-1):
			if used[i]:
				continue
			pi = sp[i]
			p = ((pi * pj) << 1) + 1
			if not passesMR(p,numMR2):
				continue
			if pow(4, pi, p) == 1:
				continue
			if pow(4, pj, p) == 1:
				continue
			bp.append(p)
			if len(bp) == 2:
				flag = False
				break
			sp.append(pj)
			used.append(True)
			used[i] = True
			flag2 = True
			break
		if not flag:
			break
		if flag2:
			continue
		sp.append(pj)
		used.append(False)
	p = bp[0]
	q = bp[1]
	if p < q:
		t = p
		p = q
		q = t
	mk = MakwaPrivateKey(p, q, 4)
	if mk.modulus.bit_length() != size:
		raise ValueError('Key generation error')
	return mk
\end{lstlisting}
Funkcja \textit{passesMR}:
\begin{lstlisting}[language=Python]
def passesMR(n,cc):
	if n<0:
		n = -n
	if n == 0:
		return True
	if n.bit_length() <= 3:
		if n == 2 or n == 3 or n == 5 or n ==7:
			return False
		else:
			return True
	if (n & (1 << 0)) == 0: # checkout BigInteger.testBit in java
		return True

	# Miller-Rabin algorithm
	nm1 = n - 1
	nm2 = nm1 - 1
	r = nm1
	s = 0
	while (r & (1 << 0)) == 0:
		s+=1
		r = r >> 1
	while cc > 0:
		cc-=1
		a = makeRandNonZero(nm2) + 1
		y = pow(a,r,n)
		if y!=1 and y!=nm1:
			for j in range(1,s):
				if y == nm1:
					break
				y = (y * y) % n
				if y == 1:
					return False
			if y!=nm1:
				return False
	return True
\end{lstlisting}
\end{document}